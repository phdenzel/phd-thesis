%% intro.tex

On May 29, 1919 during a total solar eclipse, three scientists, Eddington,
Dyson, and Davidson, tried to find out what effect, if any, is produced by a
gravitational field on the path of a light ray traversing
it~\cite{Eddington1920}.  During a thoroughly prepared expedition, they measured
the deflection of positions of stars in the well-known Hyades constellation
caused by the mass of the Sun as predicted by the theory of gravity, general
relativity.  This was the first observation of gravitational lensing, as well as
the first predicted and validated effect of general
relativity~\cite{Einstein1911}.

Effectively, gravitational lensing is entirely analogous to optical lenses. In
fact, to produce the distorted light signatures like they are observed in
gravitational lens systems, one only needs a stem and the base from a wine glass
and move it in front of a light source.
\href{https://phdenzel.github.io/zurich-lens/}{phdenzel.github.io/zurich-lens/}
provides an interactive toy example of such a situation.  In optics, the lens
comprises a glass sheet of variable thickness.  It deflects light by an amount
proportional to the local depth.  The physical process which causes the change
of direction when a ray traverses a glass lens, is called refraction.  It
describes the delay, i.e. decrease in speed of light, due to the change of media
and therefore a change in direction.  While the result of lensing is the same
for both optical and gravitational lenses, the latter causes the delay and
deflection due to the change of the felt gravitational potential as light moves
through it.  The rather rare occurrence of a perfect alignment of a background
source, a quasar for instance, and a massive foreground lens in which the imaged
source is distorted in a way such that it can be observed multiple times or even
as a ring wrapping around the lens, is called \textit{strong} gravitational
lensing.  \figref{mock-lens} demonstrates such a case with a mock lens.
%
\begin{figure}[h]%
    \centering%
    \includegraphics[width=0.49\textwidth]{m82-original}\,%
    \includegraphics[width=0.49\textwidth]{m82-lensmock}
    \caption[Mock lens image of M82]{Mock lens observation: The left image shows
      the "Cigar Galaxy" M82 cutout from
      \href{https://apod.nasa.gov/apod/ap200515.html}{APOD 2020 May 15}. The
      right image demonstrates what it could look like, if a small black hole
      for instance with 3 times the mass of the Sun would replace the Moon
      (\textcolor{red}{TODO: check calculation again, angular size of 10
      arcmin}).  The right image was generated using my lens mock code:
      \Code{lensing.js} \cite{lensing.js}.\\
      \textit{Image Credit \& Copyright: Dietmar Hager, Torsten
        Grossmann}}%
    \figlbl{mock-lens}%
\end{figure}%
%

More specifically, the deflection in gravitational lensing is of the order of
$4M/R$, where $M$ is the mass of the lens and $R$ its size.  Strong
gravitational lensing occurs when the apparent size $R/D$ of the aligned lens at
a distance $D$ is comparable to that deflection \sidecite{MagicEnv}.  In fact,
although the underlying physical process is all the same, gravitational lensing
is categorised into three types based on the observational techniques and mass
or size regimes: and \textit{micro}, \textit{strong}, and \textit{weak} lensing.
Note that in the following equation and the entire introduction of this thesis,
we assume $c=G=1$, which enables us to express distances and masses in units of
seconds, mass densities and velocity gradients in units of squared seconds etc.
%
\begin{equation}\eqlbl{lensing_types}%
  \begin{aligned}
    \text{micro} \hspace{1cm}&\frac{4M}{R} \gg \frac{R}{D}\\
    \text{strong} \hspace{1cm}&\frac{4M}{R} \gtrsim \frac{R}{D}\\
    \text{weak} \hspace{1cm}&\frac{4M}{R} < \frac{R}{D}
  \end{aligned}
\end{equation}%
%
Microlensing is in many cases due to small, compact, and massive objects such as
stars, or even exoplanets around stars.  While it technically also projects a
source in multiple images, their angular separation is typically of the order of
microarcseconds --- hence the name --- and therefore impossible to resolve with
even the most modern telescopes. Nevertheless, in microlensing the changes in
the source alignments express as changes in apparent brightness, which is
detectable over an observation period of $\sim100$ days.  Weak lensing on the
other hand, happens when the gravitational field of the lens is not strong
enough to create multiple images, and the observable effect is a distortion
which is only detectable in a statistical sense.

Most gravitational lenses lie at cosmological distances, and only very massive,
large, and mass-concentrated objects can in this case lead to strong lensing
features.  Galaxies are vast cosmic islands of stars, gas, dust, and mainly
non-luminous matter held together by gravity.  This puts them in an ideal mass
and size range to act as such strong-lensing systems.  Based on most recent
observations, it is estimated that there are more than 2 trillion galaxies in
the Universe \sidecite{Conselice16}, however so far less than a thousand
gravitational lenses have been found across various data sets.  This makes them
quite the rare beasts in comparison\dots\ at least for now.  Future space and
ground-based telescope missions however, such as the \textit{Square Kilometer
Array} (SKA), the \textit{Vera Rubin Observatory} (formerly known as
\textit{Large Synoptic Survey Telescope}, or LSST), the \textit{James Webb Space
Telescope} (JWST), and \textit{Euclid} are expected to find orders of magnitudes
more.  This is very promising for science, because, due to their very special
circumstances, lensing galaxies and their configurations can be modelled and
give otherwise unobtainable insights into galaxy structure. The task of lens
models is to describe particular shapes of the lensing galaxy's mass
distribution which produces deflections of one or more background sources which
agree with the observed lensed images.  The fact that gravity and thus lensing
is indiscriminate of the kinds of matter which cause the deflections, makes it
all the more interesting.  In this regard, gravitational lenses are often seen
as 'proof'\sidenote{Not the only proof, as explained in following sections.} for
invisible and non-luminous matter, commonly called \textit{dark matter}, without
which the observed deflections due to galaxies cannot be explained by general
relativity.

Finding connections between these matter components, how they assembled, and how
they dynamically interact within galaxies is still subject of ongoing research.
Solving galaxy formation and evolution is a crucial part in understanding the
evolution of the Universe as a whole and the physical laws driving it.  During
the last two decades, large-scale hydrodynamical simulations have yielded great
successes and managed to produce galaxy models which agree with astronomical
observations incredibly well.  In contrast, lens models still struggle to
conform with physical properties and are hard to interpret.  Some describe lens
reconstruction techniques as 'black art' \sidecite{SaasFee}, perhaps because
they produce models which are, if at all, only barely motivated by physical
processes which are believed to be key in the dynamics of galaxies. Another
reason might be the confusion seeded by very opposing opinions\sidenote{Excerpt
from \citeay{SaasFee}: "I will argue that the parametric models are all that is
needed to model lenses and that they provide a better basis for understanding
the results than non-parametric models (but the reader should be warned that if
Prasenjit Saha was writing this you would probably get a different opinion)."}
of how these lens models should be constructed.  Nevertheless, while there are
advocates for certain (and not other) approaches, they all agree that these lens
systems are, scientifically speaking, highly valuable and promise to uncover
mysteries surrounding galaxy evolution, the nature of dark matter, galaxy
substructures, and the expansion of the Universe.

The main theme of this thesis is to explore old and new ways of connecting the
three cornerstones of lensing research, \textit{lensing galaxies} from
observations, \textit{lens models} which try to reproduce observations, and
\textit{simulations} of galaxies.  The following chapters detail projects with
that aim, each designated by a project name: \Code{DELAYS}~(\chref{delays}),
\Code{FOSSIL}~(\chref{fossil}), \Code{ADLER}~(\chref{adler}), and
\Code{MATCH}~(\chref{match}).

%
\begin{figure}[h]%
    \centering%
    \includegraphics[width=0.99\textwidth]{scheme}%
    \caption[Conceptual research graph]{Conceptual graph: the outline of this
        fictional lensing system perfectly construes the current state of
        lensing research.  Simulations of galaxies have been shown to be
        physically realistic and have successfully been inferred by lens models
        in (blind) tests.  However, lens models still struggle to uniquely
        describe lensing galaxies in observations and a direct link from
        simulations to lensing galaxies has so far never even been explored
        before.  The projects in the subsequent chapters thematise these
        subjects and their connections.  }%
    \figlbl{concept}%
\end{figure}%
%

Attempts to describe lensing galaxies with models of lens systems have been made
since the very first discovery by~\sideciteay{Walsh1979} in 1979.  In
\chref{delays} and \chref*{fossil}, new techniques were developed to optimise
lens recovery --- still rather traditionally --- with time delay measurements or
stellar population synthesis models respectively.  Even with more physical
information on the lensing galaxies, the models were difficult to properly
constrain and revealed certain issues and limitations.  The subsequent project
in \chref{adler} reports on a blind test which further explores these problems
with mock lenses from large-scale hydrodynamical simulations.  Particularly hard
to solve is the issue of degeneracies, the fact that a lens configuration can be
caused by many, differently shaped lenses, which is a long-known, inherent
limitation to lens models.  It also proposes analysis techniques which can be
employed in this case to isolate these problems.  Finally, \chref{match} gives
proof-of-concept for an entirely new strategy, a lens 'matching' technique,
which provides a direct link from simulations to lensing galaxies in
observations.

The following sections introduce the subsequent chapters and cover some basics
and related topics in the schematic of \figref{concept}.  They elaborate on a
few key aspects in cosmology (\secref{exp_universe}), some basic lensing theory
(\secref{lensing}), and theoretical and observational background related to
galactic dynamics, in particular focusing on topics relevant to galaxies in
lensing systems (\secref{galaxies}).  Finally, the subsequent chapters are put
into context and shortly summarized in \secref{summary}.

%\clearpage
\section{The expanding Universe}\seclbl{exp_universe}
%% exp_universe.tex

% Discovery of the expanding universe
Like most fundamental theories in physics which helped in the present-time
understanding of our Universe and the interactions within, general relativity
was formulated in the early 20th century.  While the word was spreading of
Einstein's construct of the supposedly quasi-static Universe which involved a
"cosmic constant" to keep it so, Vesto Melvin Slipher and Edwin Hubble performed
the key measurements which provided the connection between theory and
observations.  By 1923, Slipher's hard work yielded a compilation of velocity
estimates for 41 galaxies.  Remarkably, most of those galaxies were receding
from us.  Half a decade afterwards, Edwin Hubble investigated the relation
between his distance measurements to these galaxies and their radial velocities.
Thereby, he effectively measured an apparently constant velocity gradient in
units of \Hunitsalt.  This constant was later named after him, the
\textit{Hubble constant}~\Ho.  Through the velocity gradient he realized
something, which could arguably be called the birth of modern cosmology: the
concept of an expanding Universe would explain why all galaxies are receding
from us and each other~\sidecite{Kirshner04, Hubble1929}\marginnote[1.8cm]{While
one would expect such a finding to be highly cited, Hubble's publication
officially counts 72 citations as of the time of writing.}.  

Hubble's realization was an impressive leap of thought, even more so, since the
prevalent idea of the Universe at the time was synonymous to today's picture of
our own galaxy, the Milky Way, beyond which the existence of anything else was
uncertain.  Only around 1920, astronomers started considering that what they
called nebulae were in fact extra-galactic "island universes" that is entirely
other galaxies.  Today, there are "standard" recipes for recreating and
improving upon Hubble's results by gathering distance and velocity estimates to
galaxies and other astronomical objects which are much further away.  While this
might seem like a simple task, the matter of measuring distances relates to
problems with which cosmology struggles still today.  

%%%%%%%%%%%%%%%%%%%%%%%%%%%%%%%%%%%%%%%%%%%%%%%%%%%%%%%%%%%%%%%%%%%%%%%%%%%%%%%%
\par\noindent\rule{\textwidth}{0.8pt}

% Parallax
The most "human" method of measuring distances as it utilizes the same
principles on which our eyes are based.

Has only short reach, even with the most sophisticated techniques resolve angles.

Since cosmology's main task is to measure the Universe's large-scale structures
This makes the parallax method quite The most notable consequence is Partially
name-giving to parallax second or parsec (the distance where 1 AU spans 1
arcsecond in the sky) using \ref{eq:au} and \ref{eq:arcsectan}:

\begin{equation}
    1\,\parsec = 1\,\AU \times \tan(1\,\arcsec)^{-1} \approx 10^{8}\,\lightsec
    \eqlbl{parsec}
\end{equation}


luminosity distances - $\text{D}_{\text{L}}$ vs. $\text{D}_{\text{A}}$

sound-related distances and standard sirens

Friedmann equation (a la youtube video) leading to density functions


\par\noindent\rule{\textwidth}{0.8pt}
%\clearpage
\section{Lens models}\seclbl{lensing}
%% lensing.tex

General introduction to lens theory and lens modelling.

%%%%%%%%%%%%%%%%%%%%%%%%%%%%%%%%%%%%%%%%%%%%%%%%%%%%%%%%%%%%%%%%%%%%%%%%%%%%%%%%	
\par\noindent\rule{\textwidth}{0.8pt}

The distance to the lens and the source $D_\mathrm{S}$ are always much larger
than the line-of-sight extent of the lens itself. This is what allows the
so-called \textit{thin-lens} approximation which allows the definition of a
projected surface-density distribution
%
\begin{equation}\eqlbl{match:thinlens}
  \kappa(\bm\theta) = \frac{4\pi GD_\mathrm{LS}D_\mathrm{L}}{c^2D_\mathrm{S}}
    \int \rho(\bm\theta, z)\,\mathrm{d}z
\end{equation}
%
from a mass-volume density $\rho$, where $D_\mathrm{LS}$ is the angular-diameter
distance from the lens to the source, $\bm\theta$ the lens plane and $z$ the
line-of-sight coordinate.

% Lensing galaxy models
In this context, it is very useful to think of an arbitrary surface-density
distribution in terms of multipoles:
%
\begin{equation}\eqlbl{match:multipoles}%
  \kappa(\bm\theta) = \langle\kappa_0(\theta)\rangle_{\phi} 
  + \sum^{\infty}_{m=1} \langle\kappa_{m}(\theta)\rangle_{\phi} e^{im\phi}
\end{equation}%
%
where the individual components $\langle\kappa_{m}(\theta)\rangle_{\phi}$ are
angular averages over the surface density, expressed in polar coordinates.  It
corresponds much to the idea that galaxies consist of a main body of matter and
many smaller substructures orbiting around it.  The monopole $\kappa_0$, dipole
$\kappa_1$ and quadrupole $\kappa_2$ are usually considered the most important
components.  While the monopole only leads to radial deflections, the
higher-order multipoles also include angular terms which are composed of two
parts containing the interior and exterior poles.  The angular structure in the
lens is mainly determined by three sources: (i) the luminous lens galaxy
including its stars, gas, and dust, (ii) the dark matter in the halo of the
galaxy, and (iii) the perturbations from nearby objects or objects along the
line of sight.  The most common angular term included by lens models is the
external shear which describes the exterior poles of the quadrupole.

% Lensing degeneracies
In fact, \sideciteay{Young1981} reported the first lens model which already
involved all components up to the quadrupole expressed in terms of an elliptical
mass distribution using a small set of parameters.  Already then, they
identified essentially every important lensing-related issue, and in particular
realized that there are many different mass distributions which can lead to
identical lensing configurations \sidecite{Young1981b}; these degeneracies
represent an inherent limitation to lens reconstructions \sidecite{Saha2000}.
They create many misconceptions about lens models even amongst experts, as often
times a single family of models is not able to fully describe lensing
observables \sidecite{Gomer19, Denzel20, Kochanek20, TDCOSMO4}.  To reduce or
even break degeneracies, modern lens reconstruction techniques try to constrain
their models with additional data besides the image positions such as relative
fluxes of the images, arc-like extended images, stellar photometry to describe
the dynamics of the lensing galaxies, or time delays between the images
\sidecite{Barnabe07, Ferreras07, Auger10, Bruderer15, Leier11, Leier16,
Nightingale19, Denzel20b, Shajib20, Wong19}.
%\clearpage
\section{Galaxies}\seclbl{galaxies}
%% galaxies.tex

Weighing galaxies is not easy, partially because most of their components are
invisible, dark matter being the most elusive amongst them.  However, much like
tree branches move in the wind, orbits of stars still feel the gravitational
pull from the mass inside them.  Stars in a more massive galaxy will move with
higher orbital speeds than in lower-mass galaxies; more formally, the mass $M$
inside the (circular) orbit of radius $r$ is $M(<r) \propto V^{2}r$, where $V$
is the orbital speed of the star.  Accordingly, by measuring the velocities of
stars at a particular distance from the centre, a mass estimate for the galaxy
can be calculated, which is a simple application of Newton's first law combined
with Kepler's law.  It is equivalent to assuming the galaxy is in hydrostatic
equilibrium formulated by the virial theorem 
\begin{equation}
    \begin{aligned}
        &V^{2} = \frac{GM(<r)}{r}\\
        &2E = -U,
    \end{aligned}
\end{equation}
where $E$ and $U$ are the kinetic and potential energy respectively
\sidecite{BinneyTremaine08, Zwicky1933}.  When the individual orbits of stars
cannot be resolved in {e.g.} spiral galaxies, other 'tracers' can be used such
as atomic hydrogen gas measured through 21-cm line radiation in the radio
spectrum \sidecite{vandeHulst1951, Muller1951, vandeHulst1954}, or as
perturbations in the optical wavelengths.  For ellipticals, the velocity
dispersion $\sigma$ measured by the spread of spectral absorption lines is the
observable which analogously measures their mass as $M(<r) \propto \sigma^{2}r$
\sidecite{Davies83, Schechter80, Wang20}.  The observables measured are, in most
cases, treated as dynamical equilibria, or temporal averages and therefore yield
(by implicitly assuming the ergodic theorem) 1-dimensional (1D) models such as
the rotation curve $V(r)$ \sidecite{Bosma17, MartinezMedina20, Ablimit20,
Cautun20, Bovy12}.

The endeavour of measuring velocities was and still is complicated, even within
the Milky Way.  Earth is revolving around our Sun, and the Solar System is
orbiting around the Galactic centre, which means relative motions have to be
carefully examined.  From some locations on Earth, a dense strip of starlight is
visible across the night sky which is indicative of the Galaxy's disk structure.

\begin{figure}[h]
    \includegraphics{apod080104}
    \caption[The Milky Way: APOD 2008 January
    4]{\href{https://apod.nasa.gov/apod/ap080104.html}{APOD 2008 January 4}: The
    Milky Way at 5000 meters.\\
    View on our own galaxy from within (recorded in the Chilean Andes).  The
    band of the dense collection of stars from the disk and the Galactic centre
    is partially covered by the typical extinction features due to dust
    clouds.  It indicates that the Milky Way possesses a stellar disk.\\
    \textit{Credit \& Copyright: Serge Brunier}}
    \figlbl{milkyway}
\end{figure}

From far away it is quite easy to recognise the typical morphology of other
galaxies through direct observations\sidenote{Provided the telescope has enough
angular resolution.}.  Measuring their rotational properties already becomes
increasingly difficult, deducing the shape and rotation patterns of the Milky
Way from within however is an undertaking of its own.

A seemingly random and dense distribution of stars as it appears in galaxies
should in principle collapse towards its potential well.  Like in many other
astrophysical scenarios, pressure gradients can take a stabilising role and
balance gravity.  These balancing pressures depend on different physical
processes and generally define limiting scales.  For some galaxies, e.g.
ellipticals, the stars' random motions are the dominant drivers towards
stability, for spiral galaxies it is their rotation about the disk's centre.  In
contrast to orbiting systems such as the Sun and Earth, where most of the mass
is located near the guiding centre of the orbits, the Milky Way's mass
distribution is more complex with different elements such as various forms of
hydrogen gas, dust, stars, and stellar remnants.  In general, the study of
galactic rotation through stars can yield insights not only into the galaxy's
morphology, but also into its formation history and mass composition.  

A powerful tool for this is the rotation curve $V(r)$.  It characterises the
orbital velocity as a function of distance from the Galactic centre. By
measuring how $V(r)$ behaves with radius, we can draw conclusions about the
Milky Way's size, total mass, and the distribution thereof.  A solid-body
rotation $V \propto r$ would mean that the enclosed mass ideally increases with
$r^{2}$, Keplerian orbits go as $V \propto r^{-\half}$, whereas $V \propto
\text{const.}$ is a result of the enclosed mass increasing as $r$.

Milky Way's rotation curve can be probed through its stars. Prime observable is
the radial velocity $v_{z}$, and in principle the tangential velocity $v_{t}$,
distance from Earth $d$ and longitude on the sky $l$ too.  Measurements of these
quantities can be combined to the so-called \textit{Oort's constants}
%
\begin{equation}\eqlbl{obs_oortsC}%
    \begin{aligned}%
        &A = \frac{v_{z}}{d\sin{2l}} \\
        &B = \frac{v_{t}}{d} - A\cos{2l}.
    \end{aligned}%
\end{equation}%
%
A caveat is the assumption that the stars, including the Sun, are on circular
orbits, which is only approximately true.  Moreover, it assumes the Milky Way
has a monotonically decreasing, symmetric potential.  Again, this is not
entirely true as spiral arms can introduce over-densities which manifest as
asymmetries and locally break monotonic behaviour in the potential.  Still,
within their limits the Oort's constants are very useful, because they can be
rewritten as
%
\begin{equation}\eqlbl{oortsC}
    \begin{aligned}%
        &A = -\frac{1}{2} \left[\frac{\derivd V}{\derivd r} - \frac{V_{0}}{R_{0}}\right] \\
        &B = -\frac{1}{2}\left[\frac{V_{0}}{R_{0}} + \frac{\derivd V}{\derivd r}\right].
    \end{aligned}%
\end{equation}
%
Recent measurements from the Gaia survey determined these constants with
$A=15.3\pm0.4\,\Oortsunitsalt$ and $B=-11.9\pm0.4\,\Oortsunitsalt$
\sidecite{Bovy17}.  These constants express the shear and vorticity of the disk
in the solar neighbourhood.  The shear essentially measures a deviation from
solid-body rotation, the vorticity how the angular momentum varies with small
changes in radius.  Adding both $A+B = -\frac{\derivd V}{\derivd r}$ yields the
velocity gradient, which seems to be relatively flat with $3.4\;\Oortsunitsalt$
as was expected from previous and alternative investigations.  From the velocity
gradient the mass density for the Milky Way can be written as follows (see
\sideciteay{BinneyTremaine08})
%
\begin{equation}\eqlbl{MWrho}
    \begin{aligned}
        \rho_{MW} &\quad\sim\quad \frac{(A+B)^2}{2\pi} \\
                  &\quad\sim\quad 2 \times 10^{-33}\;\sec^{-2}
    \end{aligned}
\end{equation}
%
As a rough estimate, this is of the order of the mean density of the Milky Way
within its volume, as long as the assumptions of the Oort's constants are valid.
This is obviously not the case outside the edge of the Galaxy. However, the size
of the Milky Way is not clearly known.  The issue lies in the ambiguity
regarding the definition of the edge of the Galaxy.  In the literature the
'$R_{200}$' is frequently used, sometimes also called the 'virial' radius within
which the mean density equals 200 times the cosmological critical density.
Another less back-of-the-envelope definition is the 'splashback' radius, a
caustic manifested in a drop in density or radial velocity.  At roughly half the
splashback radius an edge can be defined where virialized material has completed
at least two pericentric passages.  This radius was recently determined for the
Milky Way to be $R_{MW} \sim 290\;\mathrm{kpc}$ \sidecite{Deason20}, and seems
to define an edge where our assumptions should still hold.  Within a cubic
megaparsec the Milky Way's mean density in \eqref{MWrho} should therefore be
diluted by the corresponding volume factor $(R_{MW} / \mathrm{Mpc})^{3} \sim 2
\times 10^{-2}$, yielding

\begin{equation}\eqlbl{MWperMpc}
    \frac{M_{MW}}{\mathrm{Mpc}^{3}} \quad\sim\quad 4 \times 10^{-35}\;\sec^{-2}
\end{equation}

%%%%%%%%%%%%%%%%%%%%%%%%%%%%%%%%%%%%%%%%%%%%%%%%%%%%%%%%%%%%%%%%%%%%%%%%%%%%%%%%	
\par\noindent\rule{\textwidth}{0.8pt}

Comparing this quantity to the cosmological matter density, which is roughly
30\% of the total energy content in the Universe.

\begin{equation}
    \begin{aligned}
        \rho_{m} &=\quad 0.3\cdot\frac{3}{8\pi}\Ho^{2} \\
                 &\sim\quad 2 \times 10^{-37}\;\sec^{-2}
    \end{aligned}
\end{equation}

Comparing these two quantities yields the percentage of Milky-Way-like galaxies
per cubic megaparsec
\begin{equation}
    \frac{\rho_m}{M_{MW}/\mathrm{Mpc}^{3}} \sim 0.5\%.
\end{equation}

In $(1\;\mathrm{Gpc})^{3}$ roughly $10^7$ MW

$10^{-3}$ of sources are strongly lensed
%\clearpage
\section{Summary}\seclbl{summary}
%% summary.tex

%
\begin{equation}
    \mathcal{P}(\bm\theta) = \frac{1}{2}\theta^{2} - 2\nabla^{-2}\kappa(\bm\theta)
\end{equation}
%

%
\begin{equation}
    \tau(\bm\theta) = \mathcal{P}(\bm\theta) - \bm\beta\cdot\bm\theta
\end{equation}
%


GLASS uses point-images
%
\begin{equation}
    \nabla\tau(\bm\theta) = 0
\end{equation}
%

Using Fermat's principle
%
\begin{equation}
    \bm\beta = \nabla\mathcal{P}(\bm\theta)
\end{equation}
%
which describes the extended images.


\subsection{Inferring the Hubble constant}

\chref{delays} presents a critical study of such a measurement.  With an
analysis on 8 time-delay lenses, I determined the Hubble constant with a value
of \sidenote{\protect\textcolor{red}{TODO}: I was told that one should use 'I'
in the intro... is this really what is usually done?}
%
\begin{equation*}%
    \Ho = 71.8^{+3.8}_{-3.3}\;\Hunitsalt
\end{equation*}%
%
The value is compatible with both the early and late measurements, with a
tension of less than $1.5\sigma$.  With a precision of 4.9\%, the measurement is
unfortunately not able to contribute to the resolution of the Hubble tension.
In fact, the investigation revealed that, if a 1\% level is at all obtainable,
it would require a joint-analysis of many more time-delay measurements of
quadruply imaging lens systems (quads).  It was not the only determination of
the Hubble constant through lensing observations recently; the H0LiCOW
collaboration \sidecite{Wong19} reported on a Hubble constant of
$73.3^{+1.7}_{-1.8}\;\Hunitsalt$ from 6 time-delay lenses a year before, and the
STRIDES collaboration \sidecite{Shajib20} determined a value of
$74.2^{+2.7}_{-3.0}\;\Hunitsalt$ from a single lens.  However, while these
results were reported with higher confidence, this study explores many different
lens models and thus accounts for lensing degeneracies by construction.
Although degeneracies are often neglected in lensing studies, it is known for a
long time that a family of lens-mass distributions is able reproduce the same
observables, but still yield different values for the Hubble constant.
Therefore, studies with the aim of inferring the Hubble constant need to solve
for (ideally) all possible solutions of mass distributions in order to retrieve
a complete model.  The study produced 8000 lens-mass distributions with 1000
values for the Hubble constant.  Most interestingly, the values appeared to be
asymmetrically distributed, which might indicate that the errors on \Ho{} could
be non-Gaussian.  These results were not entirely unexpected since the
investigation was actually a continuation of ideas gathered during a
participation in a scientific blind study \sidecite{TDLMC2} in which 50
simulated time-delay lenses were analysed by several research groups.  It
discovered that current lens recovery methods are accurate to only about 6\%,
even with acclaimed precision over nearly 1\%.  In comparison with the study
presented in \chref{delays}, it became apparent that the lens simulations
considerably differed from real observations and that the uncertainties in the
inference of \Ho{} from mock lenses were higher.  Besides the obvious numerical
deviations, the radial distribution of lens images of the simulated lens set was
relatively narrow which provided only little constraints on the slope of the
density profile and consequently on \Ho.  This could mean that there exists a
limit to the accuracy on \Ho{} achievable with time-delay galaxy lenses, which
ultimately might preclude them to infer \Ho{} on a level required to resolve the
Hubble tension.  Nonetheless, gravitational lenses are excellent cosmological
probes. Especially cluster lenses do not exhibit this limitation and might still
recover the Hubble constant with sufficient precision to contribute to the
resolution of the Hubble tension.

\subsection{Lens recovery of SW05}

\subsection{Modelling tests on simulations}

\subsection{The lens-matching method}

