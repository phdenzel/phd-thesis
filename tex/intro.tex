%% intro.tex

On May 29, 1919 during a total solar eclipse, three scientists, Eddington,
Dyson, and Davidson, tried to find out what effect, if any, is produced by a
gravitational field on the path of a light ray traversing
it~\cite{Eddington1920}.  During a thoroughly prepared expedition, they measured
the deflection of positions of stars in the previously studied Hyades cluster
caused by the mass of the Sun as predicted by the theory of gravity, general
relativity.  This was the first observation of gravitational lensing, as well as
the first predicted and validated effect of general
relativity~\sidecite{Einstein1911}.

Effectively, gravitational lensing is entirely analogous to optical lenses. In
fact, to produce the distorted light signatures like they are observed in
gravitational lens systems, one only needs a stem from a wine glass and move it
in front of a light source.  A toy example of such a situation can be found
under:
\href{https://phdenzel.github.io/zurich-lens/}{phdenzel.github.io/zurich-lens/}.
In optics, the lens comprises a glass sheet of variable thickness.  It deflects
light by an amount proportional to the local thickness.  The physical process
which causes a change of direction when a ray traverses a glass lens, is called
refraction.  It describes the delay, i.e. decrease in speed of light, due to the
change of media and therefore a change in direction.  While the result of
lensing is the same for both optical and gravitational lenses, the latter causes
the delay and deflection due to the change of the felt gravitational potential
as light moves through it.  The rather rare occurrence of a perfect alignment of
a background source, e.~g. a quasar, and a massive foreground object in which
the imaged source is distorted in a way such that it can be observed multiple
times or even as a ring wrapping around the lens, is called \textit{strong}
gravitational lensing.  \figref{mock-lens} demonstrates such a case with a
mock lens.
\begin{figure}[h]
    \centering
    \includegraphics[width=0.49\textwidth]{m82-original}\,%
    \includegraphics[width=0.49\textwidth]{m82-lensmock}
    \caption[Mock lens image of M82]{Mock lens observation: The left image shows
    the "Cigar Galaxy" M82 cutout from
    \href{https://apod.nasa.gov/apod/ap200515.html}{APOD 2020 May 15}. The right
    image demonstrates what it could look like if for instance a small black
    hole with a third of the Sun's mass would replace the Moon in front of M82.
    The right image was generated using my lens mock code: \Code{lensing.js}
    (\href{https://phdenzel.github.io/lensing.js/}{phdenzel.github.io/lensing.js/}).\\
    \textit{Image Credit \& Copyright: Dietmar Hager, Torsten Grossmann}}
    \figlbl{mock-lens}
\end{figure}

More specifically, the deflection in gravitational lensing is of the order of
$4M/R$, where $M$ is the mass of the lens and $R$ its size.  Strong
gravitational lensing occurs when the apparent size $R/D$ of the aligned lens at
a distance $D$ is comparable to that deflection.  In fact, although the
underlying physical process is all the same, gravitational lensing is
categorized into three types based on the observational techniques and mass or
size regimes: \textit{strong}, \textit{weak}, and \textit{micro} lensing.
\begin{align}
    \text{strong} \hspace{1cm}&\frac{4M}{R} \gtrsim \frac{R}{D}\\
    \text{weak} \hspace{1cm}&\frac{4M}{R} < \frac{R}{D}\\
    \text{micro} \hspace{1cm}&\frac{4M}{R} \gg \frac{R}{D}
\end{align}

\par\noindent\rule{\textwidth}{0.8pt}

Micro lensing is mostly used to

Analysis and research of gravitational lenses\dots

\par\noindent\rule{\textwidth}{0.8pt}

\section{The Expanding Universe}\seclbl{exp_universe}
%% exp_universe.tex

% Discovery of the expanding universe
Like most fundamental theories in physics which helped in the present-time
understanding of our Universe and the interactions within, general relativity
was formulated in the early 20th century.  While the word was spreading of
Einstein's construct of the supposedly quasi-static Universe which involved a
"cosmic constant" to keep it so, Vesto Melvin Slipher and Edwin Hubble performed
the key measurements which provided the connection between theory and
observations.  By 1923, Slipher's hard work yielded a compilation of velocity
estimates for 41 galaxies.  Remarkably, most of those galaxies were receding
from us.  Half a decade afterwards, Edwin Hubble investigated the relation
between his distance measurements to these galaxies and their radial velocities.
Thereby, he effectively measured an apparently constant velocity gradient in
units of \Hunitsalt.  This constant was later named after him, the
\textit{Hubble constant}~\Ho.  Through the velocity gradient he realized
something, which could arguably be called the birth of modern cosmology: the
concept of an expanding Universe would explain why all galaxies are receding
from us and each other~\sidecite{Kirshner04, Hubble1929}\marginnote[1.8cm]{While
one would expect such a finding to be highly cited, Hubble's publication
officially counts 72 citations as of the time of writing.}.  

Hubble's realization was an impressive leap of thought, even more so, since the
prevalent idea of the Universe at the time was synonymous to today's picture of
our own galaxy, the Milky Way, beyond which the existence of anything else was
uncertain.  Only around 1920, astronomers started considering that what they
called nebulae were in fact extra-galactic "island universes" that is entirely
other galaxies.  Today, there are "standard" recipes for recreating and
improving upon Hubble's results by gathering distance and velocity estimates to
galaxies and other astronomical objects which are much further away.  While this
might seem like a simple task, the matter of measuring distances relates to
problems with which cosmology struggles still today.  

%%%%%%%%%%%%%%%%%%%%%%%%%%%%%%%%%%%%%%%%%%%%%%%%%%%%%%%%%%%%%%%%%%%%%%%%%%%%%%%%
\par\noindent\rule{\textwidth}{0.8pt}

% Parallax
The most "human" method of measuring distances as it utilizes the same
principles on which our eyes are based.

Has only short reach, even with the most sophisticated techniques resolve angles.

Since cosmology's main task is to measure the Universe's large-scale structures
This makes the parallax method quite The most notable consequence is Partially
name-giving to parallax second or parsec (the distance where 1 AU spans 1
arcsecond in the sky) using \ref{eq:au} and \ref{eq:arcsectan}:

\begin{equation}
    1\,\parsec = 1\,\AU \times \tan(1\,\arcsec)^{-1} \approx 10^{8}\,\lightsec
    \eqlbl{parsec}
\end{equation}


luminosity distances - $\text{D}_{\text{L}}$ vs. $\text{D}_{\text{A}}$

sound-related distances and standard sirens

Friedmann equation (a la youtube video) leading to density functions


\par\noindent\rule{\textwidth}{0.8pt}

\section{The Solar Neighbourhood}\seclbl{solar_nbhood}
%% solar_nbhood.tex

Even within the Milky Way, measuring velocities and distances was and still is a
complicated endeavour.  Earth is revolving around our Sun, and the Solar System
is orbiting around the Galactic center, which means relative motions have to be
carefully examined.  From some locations on Earth, a dense strip of starlight is
visible across the night sky which is indicative of the Galaxy's disk structure.

\begin{figure}[h]
    \includegraphics{apod080104}
    \caption[The Milky Way: APOD 2008 January
    4]{\href{https://apod.nasa.gov/apod/ap080104.html}{APOD 2008 January 4}: The
    Milky Way at 5000 meters.\\
    View on our own galaxy from within (recorded in the Chilean Andes).  The
    band of the dense collection of stars from the disk and the Galactic center
    is partially covered by the typical extinction features due to dust
    clouds.  This suggest that the Milky Way possesses a stellar disk.\\
    \textit{Credit \& Copyright: Serge Brunier}}
    \figlbl{milkyway}
\end{figure}

From far away it is quite easy to recognize the typical morphology of other
galaxies through direct observations\sidenote{Provided the telescope has enough
angular resolution.}.  Measuring their rotational properties already becomes
increasingly difficult, deducing the shape and rotation patterns of the Milky
Way from within however is an undertaking of its own.

A possibly random and dense distribution of stars as it appears in galaxies
should in principle collapse towards its potential well.  Like in many other
astrophysical scenarii, pressure gradients can take a stabilizing role and
balance gravity.  These balancing pressures depend on different physical
processes and generally define limiting scales.  For some galaxies, e.g.
ellipticals, the stars' random motions are the dominant drivers towards
stability, for spiral galaxies it is their rotation about the disk's center.  In
contrast to orbiting systems such as the Sun and Earth, where most of the mass
is located near the guiding center of the orbits, the Milky Way's mass
distribution is more complex with different elements such as various forms of
hydrogen gas, dust, stars, and stellar remnants.  In general, the study of
galactic rotation through stars can yield insights not only into the galaxy's
morphology, but also into its formation history and mass composition.  One of
the most powerful tools therefor is the rotation curve $V(R)$.  It characterizes
the orbital velocity as a function of distance from the Galactic center. By
measuring how $V(R)$ behaves with radius, we can draw conclusions about the
Milky Way's size, total mass, and the distribution thereof.  A solid-body
rotation $V \propto R$ would mean that the enclosed mass ideally increases with
$R^{2}$, Keplerian orbits go as $V \propto R^{-\half}$, whereas $V \propto
\text{const.}$ is a result of the enclosed mass increasing as $R$.

Milky Way's rotation curve can be probed through its stars. Prime observable is
the radial velocity $v_{r}$, and in principle the tangential velocity $v_{t}$,
distance from Earth $d$ and longitude on the sky $l$ too.  Measurements of these
quantities can be combined to the so-called \textit{Oort's constants}
\begin{align}
    &A = \frac{v_{r}}{d\sin{2l}} \hspace{0.5cm}\nonumber\\[5pt]%
    &B = \frac{v_{t}}{d} - A\cos{2l}.
    \eqlbl{obs_oortsC}
\end{align}

A caveat is the assumption that the stars, including the Sun, are on circular
orbits, which is only approximately true.  Moreover, it assumes the Milky Way
has a monotonically decreasing, symmetric potential.  Again, this is not
entirely true as spiral arms can introduce over-densities which manifest as
asymmetries and locally break monotonic behaviour in the potential.  Still,
within their limits the Oort's constants are very useful, because they can be
rewritten as
\begin{align}
    &A = -\frac{1}{2} \left[\frac{\derivd V}{\derivd R} - \frac{V_{0}}{R_{0}}\right]\nonumber\\[5pt]%
    &B = -\frac{1}{2}\left[\frac{V_{0}}{R_{0}} + \frac{\derivd V}{\derivd R}\right].
    \eqlbl{oortsC}
\end{align}

These constants express the shear and vorticity of the disk in the solar
neighbourhood.  The shear essentially measures a deviation from solid-body
rotation, the vorticity how the angular momentum varies with small changes in
radius.  Adding both $A+B = -\frac{\derivd V}{\derivd R}$

\section{Number of lensing galaxies}\seclbl{number_lenses}
\input{tex/number_lenses}