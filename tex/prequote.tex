\begin{prequote}[20pt]{Ovid Metamorphoses, Liber XV, 179-187}
    "... ipsa quoque adsiduo labuntur tempora motu,\\
    non secus ac flumen. neque enim consistere flumen,\\
    nec levis hora potest, sed ut unda inpellitur unda,\\
    urgeturque prior veniente urgetque priorem,\\
    tempora sic fugiunt pariter pariterque sequuntur\\
    et nova sunt semper. nam quod fuit ante, relictum est,\\
    fitque, quod haut fuerat, momentaque cuncta novantur.\\
    Cernis et emensas in lucem tendere noctes,\\
    et iubar hoc nitidum nigrae succedere nocti..."\\

    While time itself glides on with ceaseless motion, not unlike a river.
    And like the stream that cannot stay, neither can the restless hour.
    Like wave impelled by wave, it hastes onward, both driven by
    and driving the next, so flee the times as they are being followed,
    and always they are anew. For what was, has ceased to be,
    and will become what never has been, and the moments are all renewed.
    You see, the departing darkness tends into the light,
    like the radiating heavenly body leads into the night.\\
\end{prequote}