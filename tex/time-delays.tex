% Paper cover page
\papertitle{TBA}
\capauthors{
    \chapterauthor[1,2]{Philipp Denzel}
    \chapterauthor[3]{Jonathan P. Coles}
    \chapterauthor[4]{Liliya L. R. Williams}
    \chapterauthor[2,1]{Prasenjit Saha}
}
\affils{
    \chapteraffil[1]{Institute for Computational Science, University of Zurich, 8057 Zurich, Switzerland}
    \chapteraffil[2]{Physics Institute, University of Zurich, 8057 Zurich, Switzerland}
    \chapteraffil[3]{Physik-Department, Technische Universit\"at M\"unchen, James-Franck-Str.~1, 85748 Garching, Germany}
    \chapteraffil[4]{School of Physics and Astronomy, University of Minnesota, 116 Church Street SE, Minneapolis, MN 55455, USA}
}

\publishedin[Reference:\\]{}
\clearpage

% Abstract
\section*{Abstract}
  %
  \noindent We present a determination of the Hubble constant from a joint,
  free-form analysis of 8 strongly, quadruply lensing systems.  In the
  concordance cosmology, we find {\Ho{}$ = 71.31^{+3.86}_{-3.61}$\,\Hunits{}}
  which is precise up to $5.24\%$ and in agreement with the latest measurements
  of the local Universe using Supernovae Type 1a calibrated by the distance
  ladder and measurements of the early Universe such as the Planck observations
  of the cosmic microwave background.  Our precision is lower compared to other
  recent measurements, because the modeling strategies reflect the systematic
  uncertainties to which lensing degeneracies can lead.  Moreover, we tested our
  free-form lens models of these systems against the current values of H$_{0}$
  from local and early Universe measurements.  We provide lensed image
  reconstructions of models constrained by these current values.
  
  This leads us to conclude that\dots
  % TODO


\clearpage
% Sections
\section{Introduction}