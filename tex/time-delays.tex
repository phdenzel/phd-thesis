% Paper cover page
\papertitle{The Hubble constant from eight time-delay galaxy lenses}
\capauthors{
    \chapterauthor[1,2]{Philipp Denzel}
    \chapterauthor[3]{Jonathan P. Coles}
    \chapterauthor[4]{Liliya L. R. Williams}
    \chapterauthor[2,1]{Prasenjit Saha}
}
\affils{
    \chapteraffil[1]{Institute for Computational Science, University of Zurich, 8057 Zurich, Switzerland}
    \chapteraffil[2]{Physics Institute, University of Zurich, 8057 Zurich, Switzerland}
    \chapteraffil[3]{Physik-Department, Technische Universit\"at M\"unchen, James-Franck-Str.~1, 85748 Garching, Germany}
    \chapteraffil[4]{School of Physics and Astronomy, University of Minnesota, 116 Church Street SE, Minneapolis, MN 55455, USA}
}

\publishedin[Reference:\\]{Denzel20b}
\clearpage


\newcommand{\Hres}{\ensuremath{71.8^{+3.9}_{-3.3}\,\Hunits{}}}
\newcommand{\HaHzres}{\ensuremath{2.33^{+0.13}_{-0.11}\,\mathrm{aHz}}}
\newcommand{\invHres}{\ensuremath{13.7^{+0.7}_{-0.7}\,\Gyrs{}}}
\newcommand{\rhocritres}{\ensuremath{5.4^{+0.6}_{-0.5}\,\GeVmcube{}}}
\newcommand{\tref}[1]{\hyperlink{tr#1}{(#1)}}
\newcommand{\tlink}[1]{\hypertarget{tr#1}{(#1)}}
\def\twidth{.59\textwidth}
\def\theight{.59\textheight}
\def\pwidth{.49\textwidth}
\def\pheight{.49\textheight}
\def\qwidth{.49\textwidth}
\def\qheight{.24\textheight}


% Abstract
\section*{Abstract}
  %
  \noindent 
  We present a determination of the Hubble constant from the joint, free-form
  analysis of 8 strongly, quadruply lensing systems.  
  In the concordance cosmology, we find $\Ho{} = \Hres{}$ with a precision of
  $4.97\%$.  This is in agreement with the latest measurements from Supernovae
  Type Ia and Planck observations of the cosmic microwave background.  
  Our precision is lower compared to these and other recent time-delay
  cosmography determinations, because our modelling strategies reflect the
  systematic uncertainties of lensing degeneracies.  We furthermore are able to
  find reasonable lensed image reconstructions by constraining to either value
  of $\Ho$ from local and early Universe measurements.  This leads us to
  conclude that current lensing constraints on $\Ho$ are not strong enough to
  break the ``Hubble tension'' problem of cosmology.


  \clearpage
  % Sections
  
  % Introduction
  
Some text here


\section{The Expanding Universe}
\seclbl{exp_universe}
%% exp_universe.tex

% Discovery of the expanding universe
Like most fundamental theories in physics which helped in the present-time
understanding of our Universe and the interactions within, general relativity
was formulated in the early 20th century.  While the word was spreading of
Einstein's construct of the supposedly quasi-static Universe which involved a
"cosmic constant" to keep it so, Vesto Melvin Slipher and Edwin Hubble performed
the key measurements which provided the connection between theory and
observations.  By 1923, Slipher's hard work yielded a compilation of velocity
estimates for 41 galaxies.  Remarkably, most of those galaxies were receding
from us.  Half a decade afterwards, Edwin Hubble investigated the relation
between his distance measurements to these galaxies and their radial velocities.
Thereby, he effectively measured an apparently constant velocity gradient in
units of \Hunitsalt.  This constant was later named after him, the
\textit{Hubble constant}~\Ho.  Through the velocity gradient he realized
something, which could arguably be called the birth of modern cosmology: the
concept of an expanding Universe would explain why all galaxies are receding
from us and each other~\sidecite{Kirshner04, Hubble1929}\marginnote[1.8cm]{While
one would expect such a finding to be highly cited, Hubble's publication
officially counts 72 citations as of the time of writing.}.  

Hubble's realization was an impressive leap of thought, even more so, since the
prevalent idea of the Universe at the time was synonymous to today's picture of
our own galaxy, the Milky Way, beyond which the existence of anything else was
uncertain.  Only around 1920, astronomers started considering that what they
called nebulae were in fact extra-galactic "island universes" that is entirely
other galaxies.  Today, there are "standard" recipes for recreating and
improving upon Hubble's results by gathering distance and velocity estimates to
galaxies and other astronomical objects which are much further away.  While this
might seem like a simple task, the matter of measuring distances relates to
problems with which cosmology struggles still today.  

%%%%%%%%%%%%%%%%%%%%%%%%%%%%%%%%%%%%%%%%%%%%%%%%%%%%%%%%%%%%%%%%%%%%%%%%%%%%%%%%
\par\noindent\rule{\textwidth}{0.8pt}

% Parallax
The most "human" method of measuring distances as it utilizes the same
principles on which our eyes are based.

Has only short reach, even with the most sophisticated techniques resolve angles.

Since cosmology's main task is to measure the Universe's large-scale structures
This makes the parallax method quite The most notable consequence is Partially
name-giving to parallax second or parsec (the distance where 1 AU spans 1
arcsecond in the sky) using \ref{eq:au} and \ref{eq:arcsectan}:

\begin{equation}
    1\,\parsec = 1\,\AU \times \tan(1\,\arcsec)^{-1} \approx 10^{8}\,\lightsec
    \eqlbl{parsec}
\end{equation}


luminosity distances - $\text{D}_{\text{L}}$ vs. $\text{D}_{\text{A}}$

sound-related distances and standard sirens

Friedmann equation (a la youtube video) leading to density functions


\par\noindent\rule{\textwidth}{0.8pt}

  
  % Systems
  \input{\home/tex/systems}
  
  % Methods
  \input{\home/tex/methods}
  
  % Results
  \input{\home/tex/results}
  
  % Conclusion
  \input{\home/tex/conclusion}

  %%%% Acknowledgements
  % 
  \section*{Acknowledgments}
  PD acknowledges support from the Swiss National Science Foundation.  This
  research is based on observations made with the NASA/ESA Hubble Space
  Telescope obtained from the Space Telescope Science Institute, which is
  operated by the Association of Universities for Research in Astronomy, Inc.,
  under NASA contract NAS 5–26555. These observations are associated with
  programs \#10158, \#15320, \#12324, \#12874, \#9744, and \#10509.

  \section*{Data availability}
  The data underlying this article are available at the STScI
  (\href{https://mast.stsci.edu/}{https://mast.stsci.edu/}; the unique
  identifiers are cited in the acknowledgements).  The derived data generated in
  this research will be shared on request to the corresponding author.
  
  %
  % Figures
  \input{\home/fig/figures}
