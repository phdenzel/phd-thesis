%% lensing.tex

General introduction to lens theory and lens modelling.

%%%%%%%%%%%%%%%%%%%%%%%%%%%%%%%%%%%%%%%%%%%%%%%%%%%%%%%%%%%%%%%%%%%%%%%%%%%%%%%%	
\par\noindent\rule{\textwidth}{0.8pt}

The distance to the lens and the source $D_\mathrm{S}$ are always much larger
than the line-of-sight extent of the lens itself. This is what allows the
so-called \textit{thin-lens} approximation which allows the definition of a
projected surface-density distribution
%
\begin{equation}\eqlbl{match:thinlens}
  \kappa(\bm\theta) = \frac{4\pi GD_\mathrm{LS}D_\mathrm{L}}{c^2D_\mathrm{S}}
    \int \rho(\bm\theta, z)\,\mathrm{d}z
\end{equation}
%
from a mass-volume density $\rho$, where $D_\mathrm{LS}$ is the angular-diameter
distance from the lens to the source, $\bm\theta$ the lens plane and $z$ the
line-of-sight coordinate.

% Lensing galaxy models
In this context, it is very useful to think of an arbitrary surface-density
distribution in terms of multipoles:
%
\begin{equation}\eqlbl{match:multipoles}%
  \kappa(\bm\theta) = \langle\kappa_0(\theta)\rangle_{\phi} 
  + \sum^{\infty}_{m=1} \langle\kappa_{m}(\theta)\rangle_{\phi} e^{im\phi}
\end{equation}%
%
where the individual components $\langle\kappa_{m}(\theta)\rangle_{\phi}$ are
angular averages over the surface density, expressed in polar coordinates.  It
corresponds much to the idea that galaxies consist of a main body of matter and
many smaller substructures orbiting around it.  The monopole $\kappa_0$, dipole
$\kappa_1$ and quadrupole $\kappa_2$ are usually considered the most important
components.  While the monopole only leads to radial deflections, the
higher-order multipoles also include angular terms which are composed of two
parts containing the interior and exterior poles.  The angular structure in the
lens is mainly determined by three sources: (i) the luminous lens galaxy
including its stars, gas, and dust, (ii) the dark matter in the halo of the
galaxy, and (iii) the perturbations from nearby objects or objects along the
line of sight.  The most common angular term included by lens models is the
external shear which describes the exterior poles of the quadrupole.

% Lensing degeneracies
In fact, \sideciteay{Young1981} reported the first lens model which already
involved all components up to the quadrupole expressed in terms of an elliptical
mass distribution using a small set of parameters.  Already then, they
identified essentially every important lensing-related issue, and in particular
realized that there are many different mass distributions which can lead to
identical lensing configurations \sidecite{Young1981b}; these degeneracies
represent an inherent limitation to lens reconstructions \sidecite{Saha2000}.
They create many misconceptions about lens models even amongst experts, as often
times a single family of models is not able to fully describe lensing
observables \sidecite{Gomer19, Denzel20, Kochanek20, TDCOSMO4}.  To reduce or
even break degeneracies, modern lens reconstruction techniques try to constrain
their models with additional data besides the image positions such as relative
fluxes of the images, arc-like extended images, stellar photometry to describe
the dynamics of the lensing galaxies, or time delays between the images
\sidecite{Barnabe07, Ferreras07, Auger10, Bruderer15, Leier11, Leier16,
Nightingale19, Denzel20b, Shajib20, Wong19}.