%% exp_universe.tex

% Discovery of the expanding universe
Like most fundamental theories in physics which helped in the present-time
understanding of our Universe and the interactions within, general relativity
was formulated in the early 20th century.  While the word was spreading of
Einstein's construct of the supposedly quasi-static Universe which involved a
'cosmic constant' to keep it so, Vesto Melvin Slipher and Edwin Hubble performed
the key measurements which provided the connection between theory and
observations.  By 1923, Slipher's hard work yielded a compilation of velocity
estimates for 41 galaxies.  Remarkably, most of those galaxies were receding
from us.  Half a decade afterwards, Edwin Hubble investigated the relation
between his distance measurements to these galaxies and their radial velocities.
Thereby, he effectively measured an apparently constant velocity gradient in
units of \Hunitsalt.  This constant was later named after him, the
\textit{Hubble constant}~\Ho.  Through this velocity gradient he realized
something, which could arguably be called the birth of modern cosmology: the
concept of an expanding Universe would explain why all galaxies are receding
from us and each other~\sidecite{Kirshner04, Hubble1929}\marginnote[1.0cm]{While
one would expect such a finding to be highly cited, Hubble's publication
interestingly counts only 72 offical citations as of the time of writing.}.  The
outward motion of galaxies resulting from the uniform expansion of the Universe
is best observable at very high distances where the local, mutual gravitational
interaction between galaxies is subdominant.  This behaviour is commonly
referred to as the \textit{Hubble flow}.

Hubble's realization was an impressive leap of thought, even more so, since the
prevalent idea of the Universe at the time was synonymous to today's picture of
our own galaxy, the Milky Way, beyond which the existence of anything else was
uncertain.  Only around 1920, astronomers started considering that what they
called nebulae were in fact extra-galactic 'island universes' that is entirely
other galaxies.  Today, there are 'standard' recipes for recreating and
improving upon Hubble's results by gathering distance and velocity estimates to
galaxies and other astronomical objects which are much further away.  While this
might seem like a simple task, the matter of measuring distances relates to
problems with which cosmology struggles still today.  

The most 'human' method of measuring distances is the \textit{parallax}.  It
essentially utilizes the same principles as the human eye.  With two points of
observation, an astronomical, stereoscopic vision is achieved from which the
distance can be estimated.  However, with even the most sophisticated
technologies reaching high angular resolution, parallax has only very little
reach.  Since the main objective in cosmology is to measure the Universe's
large-scale structures, this technique is relatively ineffective as it rarely
reaches object able to probe the Hubble flow.  Still, it is generally used as
calibration for other techniques with longer reach.  An especially powerful and
recent application of the parallax is the measurement of distances to galaxies
containing megamasers, gas clouds with water molecules which catalyse the
emission of coherent microwave radiation.  Distances to these emission points
can be determined to incredible precision with long baseline interferometers and
spectral monitors.  The most influential consequence of the parallax technique
is reflected in the distance unit 'parallax second' which is usually shortened
to \textit{parsec} and ubiquitously employed in astronomy and astrophysics.  It
is the distance where 1 astronomical unit (AU; the nominal distance of Earth to
the Sun) spans 1 arcsecond on the sky.  Using \ref{eq:au} and
\ref{eq:arcsectan}, we can write:
\begin{equation}
    1\,\parsec = 1\,\AU \times \tan(1\,\arcsec)^{-1} \approx 10^{8}\,\lightsec
    \eqlbl{parsec}
\end{equation}
 

%%%%%%%%%%%%%%%%%%%%%%%%%%%%%%%%%%%%%%%%%%%%%%%%%%%%%%%%%%%%%%%%%%%%%%%%%%%%%%%%
\par\noindent\rule{\textwidth}{0.8pt}

luminosity distances - $\text{D}_{\text{L}}$ vs. $\text{D}_{\text{A}}$

sound-related distances and standard sirens

Friedmann equation (a la youtube video) leading to density functions


\par\noindent\rule{\textwidth}{0.8pt}
%%%%%%%%%%%%%%%%%%%%%%%%%%%%%%%%%%%%%%%%%%%%%%%%%%%%%%%%%%%%%%%%%%%%%%%%%%%%%%%%
