% Paper cover page
\papertitle{A new strategy for matching observed and simulated lensing galaxies}
\capauthors{
    \chapterauthor[1,2]{Philipp Denzel}
    \chapterauthor[3]{Sampath Mukherjee}
    \chapterauthor[2,1]{Prasenjit Saha}
}
\affils{
    \chapteraffil[1]{Institute for Computational Science, University of Zurich, 8057 Zurich, Switzerland}
    \chapteraffil[2]{Physics Institute, University of Zurich, 8057 Zurich, Switzerland}
    \chapteraffil[3]{STAR Institute, Quartier Agora - All\'ee du six Ao$\hat{u}$t, 19c B-4000 Li\`ege, Belgium}
}

%\publishedin[Reference:\\]{}
\clearpage


\newcommand\SDSSJ[1]{\ifthenelse{\equal{#1}{0029}}{\textsc{J0029\textminus0055}}{}\ifthenelse{\equal{#1}{0737}}{\textsc{J0737+3216}}{}\ifthenelse{\equal{#1}{0753}}{\textsc{J0753+3416}}{}\ifthenelse{\equal{#1}{1051}}{\textsc{J1051+4439}}{}\ifthenelse{\equal{#1}{0956}}{\textsc{J0956+5100}}{}\ifthenelse{\equal{#1}{1430}}{\textsc{J1430+6104}}{}\ifthenelse{\equal{#1}{1627}}{\textsc{J1627\textminus0053}}{}\ignorespaces}%
\newcommand\roche{\mathcal{P}}
\renewcommand{\tref}[1]{\hyperlink{tr#1}{(#1)}}
\renewcommand{\tlink}[1]{\hypertarget{tr#1}{(#1)}}



\def\pwidth{.99\textwidth}
\def\qheight{.23\textheight}


% Abstract
\section*{Abstract}
\noindent The study of strong-lensing systems conventionally involves
constructing a mass distribution that can reproduce the observed
multiply-imaging properties.  Such mass reconstructions are
generically non-unique.  Here we present an alternative strategy:
instead of modelling the mass distribution, we search cosmological
galaxy-formation simulations for plausible matches.  In this paper we
test the idea on seven well-studied lenses from the SLACS survey.  For
each of these, we first pre-select a few hundred galaxies from the
EAGLE simulations, using the expected Einstein radius as an initial
criterion.  Then, for each of these pre-selected galaxies, we fit for
the source light distribution, while using MCMC for the placement and
orientation of the lensing galaxy, so as to reproduce the multiple
images and arcs.  The results indicate that the strategy is feasible,
and even yields relative posterior probabilities of two different
galaxy-formation scenarios, though these are not statistically
significant yet.  Extension to other observables, such as kinematics
and colours of the stellar population in the lensing galaxy, is
straightforward in principle, though we have not attempted it yet.
Scaling to arbitrarily large numbers of lenses also appears feasible.
This will be especially relevant for upcoming wide-field surveys,
through which the number of galaxy lenses will rise possibly a
hundredfold, which will overwhelm conventional modelling methods.

\clearpage


% Introduction

Some text here


\section{The Expanding Universe}
\seclbl{exp_universe}
%% exp_universe.tex

% Discovery of the expanding universe
Like most fundamental theories in physics which helped in the present-time
understanding of our Universe and the interactions within, general relativity
was formulated in the early 20th century.  While the word was spreading of
Einstein's construct of the supposedly quasi-static Universe which involved a
"cosmic constant" to keep it so, Vesto Melvin Slipher and Edwin Hubble performed
the key measurements which provided the connection between theory and
observations.  By 1923, Slipher's hard work yielded a compilation of velocity
estimates for 41 galaxies.  Remarkably, most of those galaxies were receding
from us.  Half a decade afterwards, Edwin Hubble investigated the relation
between his distance measurements to these galaxies and their radial velocities.
Thereby, he effectively measured an apparently constant velocity gradient in
units of \Hunitsalt.  This constant was later named after him, the
\textit{Hubble constant}~\Ho.  Through the velocity gradient he realized
something, which could arguably be called the birth of modern cosmology: the
concept of an expanding Universe would explain why all galaxies are receding
from us and each other~\sidecite{Kirshner04, Hubble1929}\marginnote[1.8cm]{While
one would expect such a finding to be highly cited, Hubble's publication
officially counts 72 citations as of the time of writing.}.  

Hubble's realization was an impressive leap of thought, even more so, since the
prevalent idea of the Universe at the time was synonymous to today's picture of
our own galaxy, the Milky Way, beyond which the existence of anything else was
uncertain.  Only around 1920, astronomers started considering that what they
called nebulae were in fact extra-galactic "island universes" that is entirely
other galaxies.  Today, there are "standard" recipes for recreating and
improving upon Hubble's results by gathering distance and velocity estimates to
galaxies and other astronomical objects which are much further away.  While this
might seem like a simple task, the matter of measuring distances relates to
problems with which cosmology struggles still today.  

%%%%%%%%%%%%%%%%%%%%%%%%%%%%%%%%%%%%%%%%%%%%%%%%%%%%%%%%%%%%%%%%%%%%%%%%%%%%%%%%
\par\noindent\rule{\textwidth}{0.8pt}

% Parallax
The most "human" method of measuring distances as it utilizes the same
principles on which our eyes are based.

Has only short reach, even with the most sophisticated techniques resolve angles.

Since cosmology's main task is to measure the Universe's large-scale structures
This makes the parallax method quite The most notable consequence is Partially
name-giving to parallax second or parsec (the distance where 1 AU spans 1
arcsecond in the sky) using \ref{eq:au} and \ref{eq:arcsectan}:

\begin{equation}
    1\,\parsec = 1\,\AU \times \tan(1\,\arcsec)^{-1} \approx 10^{8}\,\lightsec
    \eqlbl{parsec}
\end{equation}


luminosity distances - $\text{D}_{\text{L}}$ vs. $\text{D}_{\text{A}}$

sound-related distances and standard sirens

Friedmann equation (a la youtube video) leading to density functions


\par\noindent\rule{\textwidth}{0.8pt}

  
% Methods
\input{\home/tex/methods.tex}

% SEAGLE
\input{\home/tex/seagle.tex}

% Test cases
\input{\home/tex/testcases.tex}

% Results
\input{\home/tex/results.tex}

% Conclusion
\input{\home/tex/conclusion.tex}


%%%% Acknowledgements
\section*{Acknowledgments}

  We would like to thank Liliya L. R. Williams and Dominique Sluse for useful
  discussions and suggestions on how to improve this paper.

  % We also thank the anonymous referee for the constructive suggestions to
  % bring the paper to its final form.

  PD acknowledges support from the Swiss National Science Foundation.  SM
  acknowledges the funding from the European Research Council (ERC) under the
  EUs Horizon 2020 research and innovation programme (COSMICLENS; grant
  agreement no. 787886).

  This research is based on observations made with the NASA/ESA Hubble Space
  Telescope obtained from the Space Telescope Science Institute, which is
  operated by the Association of Universities for Research in Astronomy, Inc.,
  under NASA contract NAS 5–26555. These observations are associated with
  programs \#10886, \#10174, \#12210, \#10494.

\section*{Data availability}
  The data underlying this article are available at the STScI
  (\href{https://mast.stsci.edu/}{https://mast.stsci.edu/}; the unique
  identifiers are cited in the acknowledgements).  The derived data generated in
  this research will be shared on request to the corresponding author, or can be
  replicated using the open-source software available at:
  \faGithub\;\href{https://github.com/phdenzel/gleam}{https://github.com/phdenzel/gleam}.


% Figures
\input{\home/fig/figures}