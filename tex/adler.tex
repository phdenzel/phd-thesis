% Paper cover page
\papertitle{Lessons from a blind study of simulated lenses: \\image reconstructions do not always reproduce true convergence}
\capauthors{
    \chapterauthor[1,2]{Philipp Denzel}
    \chapterauthor[3]{Sampath Mukherjee}
    \chapterauthor[4]{Jonathan P. Coles}
    \chapterauthor[2,1]{Prasenjit Saha}
}
\affils{
    \chapteraffil[1]{Institute for Computational Science, University of Zurich, 8057 Zurich, Switzerland}
    \chapteraffil[2]{Physics Institute, University of Zurich, 8057 Zurich, Switzerland}
    \chapteraffil[3]{STAR Institute, Quartier Agora - All\'ee du six Ao$\hat{u}$t, 19c B-4000 Li\`ege, Belgium}
    \chapteraffil[4]{Physik-Department, Technische Universit\"at M\"unchen, James-Franck-Str.~1, 85748 Garching, Germany}
}

\publishedin{Denzel20}
\clearpage


\def\pwidth{.66\textwidth}
\def\qwidth{.49\textwidth}
\def\pheight{.19\textheight}


% Abstract
\section*{Abstract}
  %
  \noindent In the coming years, strong gravitational lens discoveries are
  expected to increase in frequency by two orders of magnitude.  Lens-modelling
  techniques are being developed to prepare for the coming massive influx of new
  lens data, and blind tests of lens reconstruction with simulated data are
  needed for validation.  In this paper we present a systematic blind study of a
  sample of 15 simulated strong gravitational lenses from the EAGLE suite of
  hydrodynamic simulations.  We model these lenses with a free-form technique
  and evaluate reconstructed mass distributions using criteria based on shape,
  orientation, and lensed image reconstruction.  Especially useful is a lensing
  analogue of the Roche potential in binary star systems, which we call the {\em
  lensing Roche potential}.  This we introduce in order to factor out the
  well-known problem of steepness or mass-sheet degeneracy.  Einstein radii are
  on average well recovered with a relative error of ${\sim}5\%$ for quads and
  ${\sim}25\%$ for doubles; the position angle of ellipticity is on average also
  reproduced well up to $\pm10\degree$, but the reconstructed mass maps tend to
  be too round and too shallow.  It is also easy to reproduce the lensed images,
  but optimizing on this criterion does not guarantee better reconstruction of
  the mass distribution.


\clearpage
% Sections

% Introduction

Some text here


\section{The Expanding Universe}
\seclbl{exp_universe}
%% exp_universe.tex

% Discovery of the expanding universe
Like most fundamental theories in physics which helped in the present-time
understanding of our Universe and the interactions within, general relativity
was formulated in the early 20th century.  While the word was spreading of
Einstein's construct of the supposedly quasi-static Universe which involved a
"cosmic constant" to keep it so, Vesto Melvin Slipher and Edwin Hubble performed
the key measurements which provided the connection between theory and
observations.  By 1923, Slipher's hard work yielded a compilation of velocity
estimates for 41 galaxies.  Remarkably, most of those galaxies were receding
from us.  Half a decade afterwards, Edwin Hubble investigated the relation
between his distance measurements to these galaxies and their radial velocities.
Thereby, he effectively measured an apparently constant velocity gradient in
units of \Hunitsalt.  This constant was later named after him, the
\textit{Hubble constant}~\Ho.  Through the velocity gradient he realized
something, which could arguably be called the birth of modern cosmology: the
concept of an expanding Universe would explain why all galaxies are receding
from us and each other~\sidecite{Kirshner04, Hubble1929}\marginnote[1.8cm]{While
one would expect such a finding to be highly cited, Hubble's publication
officially counts 72 citations as of the time of writing.}.  

Hubble's realization was an impressive leap of thought, even more so, since the
prevalent idea of the Universe at the time was synonymous to today's picture of
our own galaxy, the Milky Way, beyond which the existence of anything else was
uncertain.  Only around 1920, astronomers started considering that what they
called nebulae were in fact extra-galactic "island universes" that is entirely
other galaxies.  Today, there are "standard" recipes for recreating and
improving upon Hubble's results by gathering distance and velocity estimates to
galaxies and other astronomical objects which are much further away.  While this
might seem like a simple task, the matter of measuring distances relates to
problems with which cosmology struggles still today.  

%%%%%%%%%%%%%%%%%%%%%%%%%%%%%%%%%%%%%%%%%%%%%%%%%%%%%%%%%%%%%%%%%%%%%%%%%%%%%%%%
\par\noindent\rule{\textwidth}{0.8pt}

% Parallax
The most "human" method of measuring distances as it utilizes the same
principles on which our eyes are based.

Has only short reach, even with the most sophisticated techniques resolve angles.

Since cosmology's main task is to measure the Universe's large-scale structures
This makes the parallax method quite The most notable consequence is Partially
name-giving to parallax second or parsec (the distance where 1 AU spans 1
arcsecond in the sky) using \ref{eq:au} and \ref{eq:arcsectan}:

\begin{equation}
    1\,\parsec = 1\,\AU \times \tan(1\,\arcsec)^{-1} \approx 10^{8}\,\lightsec
    \eqlbl{parsec}
\end{equation}


luminosity distances - $\text{D}_{\text{L}}$ vs. $\text{D}_{\text{A}}$

sound-related distances and standard sirens

Friedmann equation (a la youtube video) leading to density functions


\par\noindent\rule{\textwidth}{0.8pt}


% Theory
\input{\home/tex/theory}

% % Methods
\input{\home/tex/methods}

% Results
\input{\home/tex/results}

% Conclusion
\input{\home/tex/conclusion}

%%%% Funding statement
% 
\section*{Funding statement}
PD acknowledges support from the Swiss National Science Foundation.
SM acknowledges the funding from the European Research Council (ERC)
under the EU’s Horizon 2020 research and innovation programme
(COSMICLENS;\@ grant agreement no.~787886).

%%%% Acknowledgements
% 
% \section*{Acknowledgments}
%

% Figures
\input{\home/fig/figures}
