
Galaxies changed our view of the Universe and of the nature of the fundamental
forces repeatedly.  They are tracers of the Universe's large-scale structure and
its evolution for the last 13 billion years or more.  They are also
``factories'' where cold hydrogen gas fuels the formation of stars, which leads
to the production of heavier elements, and eventually to life.  Moreover, it
appears that the luminous matter in galaxies only accounts for 1 to 5\% of their
entire mass, and the rest is seemingly invisible.  The exact nature of their
``dark'' components is perhaps the gravest enigma of modern science.  Exploring
and understanding galaxies, their formation, and evolution is therefore of
paramount interest to cosmology, particle physics, astronomy, and astrophysics.

In this thesis, we explore how models of galaxies can be incorporated with
observations of the strong gravitational lensing phenomenon in order to test
galaxy formation theories.  Despite the unique circumstances leading to lensing
systems which provide the opportunity to unravel otherwise hidden properties of
their lensing galaxies, lens-modelling attempts have yet to consistently yield
quantifiable constraints on galaxy formation scenarios.  Conventional
lens-modelling techniques use a set of parameters to describe the form of
lensing galaxies.  Here, we first examine free-form lens models and apply
measured time delays of lensed quasars in order to constrain the cosmological
parameter for the rate of cosmic expansion, the Hubble constant, but also to
optimize the lens models themselves.  Furthermore, a particularly interesting
study of an extraordinary lensing system is presented for which lens models
indicate a distinct type of galaxy, a fossil group galaxy, which is considered
the final phase of a group galaxy's evolution.  We then investigate drawbacks
and issues of such free-form lens models in a blind-study with simulated lenses.
While conventional techniques conveniently and efficiently generate models, they
remain rather simplified.  Galaxy-formation simulations on the other hand, are
highly tuned to produce realistic galaxy models from first principles.  For this
reason, we tested the plausibility of utilizing such simulations for a direct
comparison to lensing data.  At last, we propose a novel strategy which
efficiently matches projected galaxy models from simulations to lensing
observations and evaluates relative posterior probabilities of the underlying
galaxy formation scenarios.  In future, such methods are essential for upcoming
wide-field surveys, as they will increase the number of galaxy lenses, and thus
the workload, possibly a hundredfold.
