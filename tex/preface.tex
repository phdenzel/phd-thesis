Some witty explanation on the style of this thesis...  maybe also elaborate on
the Russian style of writing only an introduction followed by the technical
stuff which contains all the papers.   

As a consequence of being a Prasenjit's PhD student, one learns to realize that
units are a tool to point out relations between different physical observables,
and that 1 solar mass and 5$\mathrm{\mu sec}$ can actually describe the same
thing.  Similarly, 1 Astronomical Unit has 500 seconds, and 1 year has pretty
much $2000\pi$ times more.  As soon as those relations become clear, it is
natural to forget about the not-so-important G's and c's and one may even dare
to set them to 1.  I tried my best to avoid it, but if at some point in thesis
the units of an equation or a statement seem to be missing, I refer to the
Appendix which might list some of the keys to solve the puzzle.