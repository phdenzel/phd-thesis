Some witty explanation on the style of this thesis...  maybe also elaborate on
the Russian style of writing only an introduction followed by the technical
stuff which contains all the papers.   

Being a Prasenjit's PhD student, one eventually learns to realize that units are
a tool to point out relations between different physical observables, and that 1
solar mass and 5 microseconds can actually describe the same thing.  Similarly,
1 Astronomical Unit has 500 seconds, and 1 year has pretty much $20'000\pi$
times more.  As soon as those relations become clear, it is natural to forget
about the not-so-important G's and c's and one may even dare to set them 1.  I
tried my best to avoid it, but if at some point in the thesis the units of an
equation or a statement seem to be missing (usually to emphasize the important
quantities), I refer to the Appendix which might list some of the keys to solve
the puzzle.

Finally, I would like to thank a couple of dear friends and colleagues.  First
of all to Prasenjit Saha for all his help.  He always was the catalyst and
supporter of the more creative ideas, and convinced me that the unconventional
paths taken are much more fun, especially if they improve and motivate change in
old-fashioned and outdated matters.  

Sampath Mukherjee, Jonathan Coles, and Liliya L. R. Williams.

Next, to Rafael Souza Lima and Tomas Tamfal
for their camaraderie.  They were the greatest office mates, and were always up
for interesting scientific musings.  (We especially enjoyed talking about the
warped side of the Universe, and speculations that the Universe and all its
objects and phenomena could be entirely described by the curvature of
space-time.)  My time in the Institute for Computational Science was made
engaging, enjoyable, and fun by so many others as well. I owe thanks to each one
of them.  You know who you are!  But my biggest thanks go to my parents and my
brother. Without their support in all aspects, my academic journey would simply
not have been possible.