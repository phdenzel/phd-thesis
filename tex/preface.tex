\small
This thesis consists of a long introduction which should convey background
information and a good summary of all following chapters, and connect them in a
single line-of-thought.  The subsequent chapters, describe individual projects
(with designated code names) which were or are to be published in scientific
journals.

In order to avoid confusion and arguments about mathematical correctness, I here
emphasize that throughout the text, I tried my best to keep consistency: for
vectors I used bold notation, and since in lensing theory most things are
calculated on 2D surfaces, the nabla operator $\nabla$ denotes gradients on
those surfaces.  Likewise, the inverse Laplace operator denoted $\nabla^{-2}$,
describes the integral form of a Laplace equation (and formally corresponds to a
convolution with a ``free-space'' Green's function).

Being Prasenjit Saha's PhD student, one eventually learns to realize that
units are tools to point out relations between different physical observables,
and that 1 solar mass and 5 microseconds can sometimes express the same concept.
Similarly, 1 Astronomical Unit has 500 seconds, and 1 year has pretty much
$20'000\pi$ times more.  As soon as those relations become implicit, it is
natural to forget about the not-so-important G's and c's and one may even dare
to set them 1.  I tried my best to make my mathematical formulations and
estimates easy to read, but if at some point in the thesis the units of an
equation or a statement seem to have mysteriously vanished (usually on purpose
to emphasize the important quantities), I refer to the Appendix which might list
some of the keys to solve the puzzle.

Finally, I would like to thank a couple of dear friends and colleagues.  First
of all to Prasenjit Saha for all his help and guidance.  He always was a
catalyst and supporter of the more creative ideas, and convinced me that the
unconventional paths taken are much more fun, especially if they improve and
motivate change in old-fashioned and outdated matters.  To Sampath Mukherjee,
for all his support and supply of lens-friendly galaxy-formation scenarios.  To
Jonathan Coles, who always had useful advice when it was needed.  To Liliya L.
R. Williams, for lending me her expertise in all things lensing, even if
sometimes it was very early in the day for her.  Also to Ignacio Ferreras, who
taught me how to decipher stellar light and bring some colour to my lens maps.
Next, to Rafael Souza Lima and Tomas Tamfal for their camaraderie.  They were
the greatest office mates, and were always up for interesting scientific
musings  (we especially enjoyed talking about the warped side of the Universe
and grand unification theories).  My time in the Institute for Computational
Science was made engaging, enjoyable, and fun by so many others as well. I owe
thanks to each one of them.  But my biggest thanks go to my parents and my
brother.  Without their support in all aspects, my academic journey would simply
have not been possible.
\normalsize
