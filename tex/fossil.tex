% Paper cover page
\papertitle{The lens SW05 J143454.4+522850: a fossil group at redshift 0.6?}
\capauthors{
    \chapterauthor[1,2]{Philipp Denzel}
    \chapterauthor[2,1]{Prasenjit Saha}
}
\collabs{
    \chapterauthor[3]{Xanthe Gwyn Palmer}
    \chapterauthor[1]{Onur \c{C}atmabacak}
    \chapterauthor[4]{Jonathan Coles}
    \chapterauthor[5]{Claude Cornen}
    \chapterauthor[6]{Ignacio Ferreras}
    \chapterauthor[1]{Robert Feldmann}
    \chapterauthor[2]{Rafael K\"ueng}
    \chapterauthor[7]{Dominik Leier}
    \chapterauthor[3]{Aprajita Verma}


}
\affils{
    \chapteraffil[1]{Institute for Computational Science, University of Zurich, 8057 Zurich, Switzerland}
    \chapteraffil[2]{Physics Institute, University of Zurich, 8057 Zurich, Switzerland}
    \chapteraffil[3]{Sub-department of Astrophysics, University of Oxford, Denys Wilkinson Building, Keble Road, Oxford, OX1 3RH, UK}
    \chapteraffil[4]{Physik-Department, Technische Universit\"at M\"unchen, James-Franck-Str.~1, 85748 Garching, Germany}
    \chapteraffil[5]{Zooniverse, c/o Astrophysics Department, University of Oxford, Oxford OX1~3RH, UK}
    \chapteraffil[6]{Mullard Space Science Laboratory, University College London, Holmbury St Mary, Dorking, Surrey RH5 6NT, UK}
    \chapteraffil[7]{DB, Germany}
}

\publishedin[To be submitted to:\\\textit{Monthly Notices of the Royal Astronomical Society}]{}
\clearpage

\newcommand{\SW}{Space\,Warps\xspace}
\def\pwidth{.99\textwidth}
\def\qwidth{.49\textwidth}
\def\qqwidth{.57\textwidth}

% Abstract
\section*{Abstract}
\noindent Fossil groups are considered to be the end product of natural galaxy
group evolution in which group members sink towards the center of the
gravitational potential due to dynamical friction, and finally merge into a
single, massive, and (x-ray) bright elliptical.  Since gravitational lensing
depends on the mass of a foreground object, its mass concentration, and distance
to the observer, we can expect lensing effects of such fossil groups to be
particularly strong.

\noindent In this paper, we consider the exceptional system J143454.4$+$522850,
labelled SW05 by the \SW~discovery service.  We combine lens models with
population-synthesis models for the stellar mass to separate the total mass of
the lens into the two main components in galaxies, stars and dark matter,
thereby producing dark matter-maps for the lens.  The observed enclosed mass
profiles were additionally compared with state-of-the-art galaxy formation
simulations, which shows interesting differences between observation and
simulations.  These differences, as well as the lens model on itself, finally
lead us to the conclusion that SW05 may be a fossil group.  It would represent
the first direct comparison of a fossil groups' dark matter halo with its
stellar component in the center.  We finally put forth the idea of a diagnostic
for the identification of fossil groups through gravitational lensing by looking
for the lack of certain features in the lens models.
%

\clearpage
% Sections
  
% Introduction

Some text here


\section{The Expanding Universe}
\seclbl{exp_universe}
%% exp_universe.tex

% Discovery of the expanding universe
Like most fundamental theories in physics which helped in the present-time
understanding of our Universe and the interactions within, general relativity
was formulated in the early 20th century.  While the word was spreading of
Einstein's construct of the supposedly quasi-static Universe which involved a
"cosmic constant" to keep it so, Vesto Melvin Slipher and Edwin Hubble performed
the key measurements which provided the connection between theory and
observations.  By 1923, Slipher's hard work yielded a compilation of velocity
estimates for 41 galaxies.  Remarkably, most of those galaxies were receding
from us.  Half a decade afterwards, Edwin Hubble investigated the relation
between his distance measurements to these galaxies and their radial velocities.
Thereby, he effectively measured an apparently constant velocity gradient in
units of \Hunitsalt.  This constant was later named after him, the
\textit{Hubble constant}~\Ho.  Through the velocity gradient he realized
something, which could arguably be called the birth of modern cosmology: the
concept of an expanding Universe would explain why all galaxies are receding
from us and each other~\sidecite{Kirshner04, Hubble1929}\marginnote[1.8cm]{While
one would expect such a finding to be highly cited, Hubble's publication
officially counts 72 citations as of the time of writing.}.  

Hubble's realization was an impressive leap of thought, even more so, since the
prevalent idea of the Universe at the time was synonymous to today's picture of
our own galaxy, the Milky Way, beyond which the existence of anything else was
uncertain.  Only around 1920, astronomers started considering that what they
called nebulae were in fact extra-galactic "island universes" that is entirely
other galaxies.  Today, there are "standard" recipes for recreating and
improving upon Hubble's results by gathering distance and velocity estimates to
galaxies and other astronomical objects which are much further away.  While this
might seem like a simple task, the matter of measuring distances relates to
problems with which cosmology struggles still today.  

%%%%%%%%%%%%%%%%%%%%%%%%%%%%%%%%%%%%%%%%%%%%%%%%%%%%%%%%%%%%%%%%%%%%%%%%%%%%%%%%
\par\noindent\rule{\textwidth}{0.8pt}

% Parallax
The most "human" method of measuring distances as it utilizes the same
principles on which our eyes are based.

Has only short reach, even with the most sophisticated techniques resolve angles.

Since cosmology's main task is to measure the Universe's large-scale structures
This makes the parallax method quite The most notable consequence is Partially
name-giving to parallax second or parsec (the distance where 1 AU spans 1
arcsecond in the sky) using \ref{eq:au} and \ref{eq:arcsectan}:

\begin{equation}
    1\,\parsec = 1\,\AU \times \tan(1\,\arcsec)^{-1} \approx 10^{8}\,\lightsec
    \eqlbl{parsec}
\end{equation}


luminosity distances - $\text{D}_{\text{L}}$ vs. $\text{D}_{\text{A}}$

sound-related distances and standard sirens

Friedmann equation (a la youtube video) leading to density functions


\par\noindent\rule{\textwidth}{0.8pt}

  
% System
\input{\home/tex/system}
  
% Lens mass model
\input{\home/tex/lensmass}
  
% Stellar mass model
\input{\home/tex/stelmass}
  
% Comparison to simulations
\input{\home/tex/simul}

% Conclusion
\input{\home/tex/disqus}

%%%% Acknowledgements
% 
\section*{Acknowledgments}
PhD acknowledges support from the Swiss National Science Foundation.