%% summary.tex

%
\begin{equation}
    \mathcal{P}(\bm\theta) = \frac{1}{2}\theta^{2} - 2\nabla^{-2}\kappa(\bm\theta)
\end{equation}
%

%
\begin{equation}
    \tau(\bm\theta) = \mathcal{P}(\bm\theta) - \bm\beta\cdot\bm\theta
\end{equation}
%


GLASS uses point-images
%
\begin{equation}
    \nabla\tau(\bm\theta) = 0
\end{equation}
%

Using Fermat's principle
%
\begin{equation}
    \bm\beta = \nabla\mathcal{P}(\bm\theta)
\end{equation}
%
which describes the extended images.


\subsection{Inferring the Hubble constant}

\chref{delays} presents a critical study of such a measurement.  With an
analysis on 8 time-delay lenses, I determined the Hubble constant with a value
of \sidenote{\protect\textcolor{red}{TODO}: I was told that one should use 'I'
in the intro... is this really what is usually done?}
%
\begin{equation*}%
    \Ho = 71.8^{+3.8}_{-3.3}\;\Hunitsalt
\end{equation*}%
%
The value is compatible with both the early and late measurements, with a
tension of less than $1.5\sigma$.  With a precision of 4.9\%, the measurement is
unfortunately not able to contribute to the resolution of the Hubble tension.
In fact, the investigation revealed that, if a 1\% level is at all obtainable,
it would require a joint-analysis of many more time-delay measurements of
quadruply imaging lens systems (quads).  It was not the only determination of
the Hubble constant through lensing observations recently; the H0LiCOW
collaboration \sidecite{Wong19} reported on a Hubble constant of
$73.3^{+1.7}_{-1.8}\;\Hunitsalt$ from 6 time-delay lenses a year before, and the
STRIDES collaboration \sidecite{Shajib20} determined a value of
$74.2^{+2.7}_{-3.0}\;\Hunitsalt$ from a single lens.  However, while these
results were reported with higher confidence, this study explores many different
lens models and thus accounts for lensing degeneracies by construction.
Although degeneracies are often neglected in lensing studies, it is known for a
long time that a family of lens-mass distributions is able reproduce the same
observables, but still yield different values for the Hubble constant.
Therefore, studies with the aim of inferring the Hubble constant need to solve
for (ideally) all possible solutions of mass distributions in order to retrieve
a complete model.  The study produced 8000 lens-mass distributions with 1000
values for the Hubble constant.  Most interestingly, the values appeared to be
asymmetrically distributed, which might indicate that the errors on \Ho{} could
be non-Gaussian.  These results were not entirely unexpected since the
investigation was actually a continuation of ideas gathered during a
participation in a scientific blind study \sidecite{TDLMC2} in which 50
simulated time-delay lenses were analysed by several research groups.  It
discovered that current lens recovery methods are accurate to only about 6\%,
even with acclaimed precision over nearly 1\%.  In comparison with the study
presented in \chref{delays}, it became apparent that the lens simulations
considerably differed from real observations and that the uncertainties in the
inference of \Ho{} from mock lenses were higher.  Besides the obvious numerical
deviations, the radial distribution of lens images of the simulated lens set was
relatively narrow which provided only little constraints on the slope of the
density profile and consequently on \Ho.  This could mean that there exists a
limit to the accuracy on \Ho{} achievable with time-delay galaxy lenses, which
ultimately might preclude them to infer \Ho{} on a level required to resolve the
Hubble tension.  Nonetheless, gravitational lenses are excellent cosmological
probes. Especially cluster lenses do not exhibit this limitation and might still
recover the Hubble constant with sufficient precision to contribute to the
resolution of the Hubble tension.

\subsection{Lens recovery of SW05}

\subsection{Modelling tests on simulations}

\subsection{The lens-matching method}
